\documentclass{article}
\usepackage[utf8]{inputenc}
\usepackage{polski}
\usepackage{graphicx}
\usepackage[polish]{babel}
\usepackage[T1]{fontenc}
\usepackage{pgfplots}

\title{Sprawozdanie - Spadek swobodny}
\author{Julia Sujka}
\date{December 2022}

\begin{document}

\maketitle

\section{Wprowadzenie teoretyczne}
Spadek swobodny albo spadanie swobodne jest to ruch ciała w polu grawitacyjnym upuszczonego z
pewnej wysokości na ziemię. Spadek swobodny jest przykładem ruchu jednostajnie przyspieszonego
prostoliniowego z przyspieszeniem $a=g=const$. Przyspieszenie to nazywamy przyspieszeniem
ziemskim i oznaczamy je literą $g$. Jeżeli spadek ma miejsce z małej wysokości w pobliżu powierzchni
Ziemi i dotyczy ciała o stosunkowo dużej gęstości i aerodynamicznym kształcie (np. kuli), wówczas
ruch takiego ciała można z dobrym przybliżeniem traktować jak ruch jednostajnie przyspieszony z
przyspieszeniem ziemskim g bez prędkości początkowej. Ruch ten opisuje kinematyczne równanie
ruchu w postaci:
\\
\\
\begin{equation}
h(t)=h_0=\frac{gt^2}{2}
\end{equation}\\
\\
gdzie: \\
$h(t)$ – wysokość, na jakiej znajduje się ciało po czasie t \\
$h_0$ - wysokość z jakiej spada ciało; \\
$t$ - czas spadku;\\
$g$ - wartość przyspieszenia ziemskiego.\\

\newpage

\section{Opis eksperymentu}
\begin{center}
\begin{figure}[hp]
\centering
\includegraphics[scale=0.45]{pic.png}
\caption{Schemat doświadczenia}
\label{fig:schemat}
\end{figure}
\end{center}
Jak widać na Rysunku \ref{fig:schemat} kulkę zrzucono z 20 metrów.

\section{Wyniki pomiarów}
\begin{table}[hp]
\caption{Spadek swobodny}
\label{tab: Spadek_swobodny}
\begin{center}
\begin{tabular}{|l|l|l|l|l|}
\hline Lp. & $t[s]$ & $s[m]$ & $delta$ $s[m]$ & $2s[m]$\\
\hline
1 & 0,100 & 0,049 & -0,708 & -0,659\\\hline
2 & 0,200 & 0,196 & 3,513 & 3,709\\\hline
3 & 0,300 & 0,441 & -4,191 & -3,750\\\hline
4 & 0,400 & 0,784 & 2,181 & 2,965\\\hline
5 & 0,500 & 1,225 & 2,646 & 3,871\\\hline
6 & 0,600 & 1,764 & 1,869 & 3,633\\\hline
7 & 0,700 & 2,401 & 3,001 & 5,402\\\hline
8 & 0,800 & 3,136 & 1,919 & 5,055\\\hline
9 & 0,900 & 3,969 & -0,862 & 3,107\\\hline
10 & 1,000 & 4,900 & 1,135 & 6,035\\\hline
11 & 1,100 & 5,929 & -1,704 & 4,225\\\hline
12 & 1,200 & 7,056 & 0,212 & 7,268\\\hline
\end{tabular}
\end{center}
\end{table}

\begin{tikzpicture}
\begin{axis}[
    axis lines = left,
    xlabel = \(Czas\),
    ylabel = {\(Droga\)},
]
%Below the red parabola is defined
\addplot [
    domain=0.0:10.00, 
    samples=100, 
    color=red,
]
{9.81*x^2 * 0.5};
\addlegendentry{\(9.81*x^2 * 0.5\)}
\addplot[
    only marks,
    scatter,
    mark size=0.5pt]
table[meta=czas]
{spadek swobodny.csv};

\end{axis}
\end{tikzpicture}


\newpage
\section{Wnioski}
Otrzymana wartość przyspieszenia ziemskiego różni się od znanej nam wartości 9.81 $\frac{m}{s^2}$, ale
pamiętajmy, że nieodłącznym czynnikiem każdego doświadczenia są błędy pomiarowe. Jeśli chodzi o
pomiar czasu, to możemy przyjąć, że błędy były niewielkie, ale za to spore błędy pojawiły się przy
pomiarze drogi

\end{document}
