# Technologie-informacyjne

\documentclass{article}
\usepackage[utf8]{inputenc}
\usepackage{polski}
\usepackage{graphicx}
\usepackage[polish]{babel}
\usepackage[T1]{fontenc}

\title{Sprawozdanie - Spadek swobodny}
\author{Julia Sujka}
\date{December 2022}

\begin{document}

\maketitle

\section{Wprowadzenie teoretyczne}
Spadek swobodny albo spadanie swobodne jest to ruch ciała w polu grawitacyjnym upuszczonego z
pewnej wysokości na ziemię. Spadek swobodny jest przykładem ruchu jednostajnie przyspieszonego
prostoliniowego z przyspieszeniem $a=g=const$. Przyspieszenie to nazywamy przyspieszeniem
ziemskim i oznaczamy je literą $g$. Jeżeli spadek ma miejsce z małej wysokości w pobliżu powierzchni
Ziemi i dotyczy ciała o stosunkowo dużej gęstości i aerodynamicznym kształcie (np. kuli), wówczas
ruch takiego ciała można z dobrym przybliżeniem traktować jak ruch jednostajnie przyspieszony z
przyspieszeniem ziemskim g bez prędkości początkowej. Ruch ten opisuje kinematyczne równanie
ruchu w postaci:
\\
\\
\begin{equation}
h(t)=h_0=\frac{gt^2}{2}
\end{equation}\\
\\
gdzie: \\
$h(t)$ – wysokość, na jakiej znajduje się ciało po czasie t \\
$h_0$ - wysokość z jakiej spada ciało; \\
$t$ - czas spadku;\\
$g$ - wartość przyspieszenia ziemskiego.\\

\newpage

\section{Opis eksperymentu}
\begin{center}
\begin{figure}[hp]
\centering
\includegraphics[scale=0.45]{pic.png}
\caption{Schemat doświadczenia}
\label{fig:schemat}
\end{figure}
\end{center}
Jak widać na Rysunku \ref{fig:schemat} kulkę zrzucono z 20 metrów.

\section{Wyniki pomiarów}
\begin{table}
\caption{Spadek swobodny}
\label{tab: Spadek_swobodny}
\begin{center}
\begin{tabular}{|c||1|1|1|1|}
\hline Lp. & $t[s]$ & $s[m]$ &
\end{tabular}
\end{center}

\end{table}
\section{Wnioski}
Otrzymana wartość przyspieszenia ziemskiego różni się od znanej nam wartości 9.81 $\frac{m}{s^2}$, ale
pamiętajmy, że nieodłącznym czynnikiem każdego doświadczenia są błędy pomiarowe. Jeśli chodzi o
pomiar czasu, to możemy przyjąć, że błędy były niewielkie, ale za to spore błędy pojawiły się przy
pomiarze drogi

\end{document}
